\documentclass[12pt,a4paper]{article}
% Trabalho de Política I
% Pedro T. R. Pinheiro (8983332)
\usepackage[utf8]{inputenc}
\usepackage[portuguese]{babel}
\usepackage{parskip}
\usepackage{setspace}
\usepackage[left=3.00cm, right=2.00cm, top=3.00cm, bottom=2.00cm]{geometry}
\usepackage{scrextend}

\newenvironment{citac}
{
	\begin{addmargin}
		[4cm]{1em} \footnotesize}{\normalfont 
	\end{addmargin}
}

\author{Pedro T. R. Pinheiro\footnote{Aluno pelo departamento de filosofia 
da FFLCH/USP, de número USP 8983332. }}
\title{Foucault e o Cárcere\footnote{
	Trabalho desenvolvido para a matéria FLF0388 - Ética e Filosofia 
	Política I, ministrada pela Profª. Drª. Silvana Ramos. }}

\begin{document}
	\maketitle
	
	\onehalfspacing
	\setlength{\parskip}{0.5cm}
	
	Poder, criminoso, condenado. Noutros tempos, fora mais simples a 
	equação. Em sua obra \textit{Vigiar e Punir}, M. Foucault postula 
	que haja ao menos três modelos penais possíveis, e este era apenas 
	o primeiro deles. Com o avanço da técnica e da ciência em tempos de 
	iluminismo, esta modelo, um mero expositor do Leviatã em toda sua 
	potência, substitui-se, transforma-se. Neste trabalho, explora-se 
	um pouco este desenvolvimento com ênfase no terceiro e último 
	modelo. Nos tempos de Foucault, a última palavra em termos de 
	``tecnologias de poder''. 

	Sobre estes três modelos, cita-se: 
	
	\begin{citac}
		[...] pode-se dizer que nos encontramos no fim do século XVIII 
		diante de três maneiras de organizar o poder de punir. [...] 
		Esquematizando muito, poderíamos dizer que, no poder 
		monárquico, a punição é um cerimonial da soberania [...]. No 
		projeto dos juristas reformadores, a punição é um processo 
		para requalificar os indivíduos como sujeitos de direito 
		[...]. Enfim, no projeto de instituição carcerária que se 
		elabora, a punição é uma técnica de coerção dos indivíduos. 
		[...] O soberano e sua força, o corpo social, o aparelho 
		administrativo. [...] São três modalidades de acordo com 
		as quais se exerce o poder de punir. Três tecnologias de poder. 
		(FOUCAULT, 1996)
	\end{citac}

	Ao delinear uma \emph{genealogia} da evolução dos sistemas penais, 
	Foucault já de saída deixa bastante clara a sua posição: ainda que 
	operem de maneiras distintas, todas tem como objetivo a consolidação 
	de um poder. No sistema penal ``tradicional'', a punição existe 
	praticamente como uma forma de represália dos detentores do poder. 
	Enquanto, nos outros dois sistemas, há um viés em direção da 
	disciplina como garantidora do poder. Grosso modo, é isto. Contudo, 
	quando se fala nos últimos dois modelos, o próprio autor afirma: 
	\newpage % ugh
	\begin{citac}
		As outras se referem, ambas, a uma concepção preventiva, 
		utilitária, corretiva de um direito de punir que pertenceria 
		à sociedade inteira; mas são muito diferentes entre si, ao 
		nível dos dispositivos que esboçam. (FOUCAULT, 1996)
	\end{citac}

	Tratemos, então, destes modelos. O primeiro deles, obra de diversos 
	juristas, tais como Dupaty, Lacratelle e, notavelmente, Beccaria, 
	é praticamente um resultado direto do \textit{zeitgeist} iluminista. 

	% COISAS A SE FAZER NESTA PORRA, AINDA

	%-> tratar do terceiro treco

	%-> prosear um pouco sobre poder/saber

	%-> se precisar, dissertar um pouco falando de "tecnologia"

	%-> comentar sobre a circularidade do texto

	%-> tentar responder à pergunta da Silvana

\end{document}
