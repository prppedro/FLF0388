\documentclass[12pt,a4paper]{article}
% Trabalho de Política I
% Pedro T. R. Pinheiro (8983332)
\usepackage[utf8]{inputenc}
\usepackage[portuguese]{babel}
\usepackage{parskip}
\usepackage{setspace}
\usepackage[left=3.00cm, right=2.00cm, top=3.00cm, bottom=2.00cm]{geometry}
\usepackage{scrextend}

\newenvironment{citac}
{
	\begin{addmargin}
		[4cm]{1em} \footnotesize}{\normalfont 
	\end{addmargin}
	% Para exportar para RTF
%	\footnotesize }{ \normalfont
}

\author{Pedro T. R. Pinheiro\footnote{Aluno pelo departamento de filosofia 
da FFLCH/USP, de número USP 8983332. }}
\title{Foucault e o Cárcere\footnote{
	Trabalho desenvolvido para a matéria FLF0388 - Ética e Filosofia 
	Política I, ministrada pela Profª. Drª. Silvana Ramos. }}
\date{5 de dezembro de 2019}

\begin{document}
	\maketitle
	
	\onehalfspacing
	\setlength{\parskip}{0.5cm}
	
	Poder, criminoso, condenado. Noutros tempos, fora mais simples a 
	equação. Em sua obra \textit{Vigiar e Punir}, M. Foucault postula 
	que haja ao menos três modelos penais possíveis, e este era apenas 
	o primeiro deles. Com o avanço da técnica e da ciência em tempos de 
	iluminismo, esta modelo, um mero expositor do Leviatã em toda sua 
	potência, substitui-se, transforma-se. Neste trabalho, explora-se 
	um pouco este desenvolvimento com ênfase no terceiro e último 
	modelo e, sobretudo, como ele parece ter se estabelecido como 
	\textit{state-of-the-art} no que diz respeito ao controle social. 

	Sobre estes três modelos, cita-se: 
	
	\begin{citac}
		[...] pode-se dizer que nos encontramos no fim do século XVIII 
		diante de três maneiras de organizar o poder de punir. [...] 
		Esquematizando muito, poderíamos dizer que, no poder 
		monárquico, a punição é um cerimonial da soberania [...]. No 
		projeto dos juristas reformadores, a punição é um processo 
		para requalificar os indivíduos como sujeitos de direito 
		[...]. Enfim, no projeto de instituição carcerária que se 
		elabora, a punição é uma técnica de coerção dos indivíduos. 
		[...] O soberano e sua força, o corpo social, o aparelho 
		administrativo. [...] São três modalidades de acordo com 
		as quais se exerce o poder de punir. Três tecnologias de poder. 
		(FOUCAULT, 1996)
	\end{citac}

	Ao delinear uma \emph{genealogia} da evolução dos sistemas penais, 
	Foucault já de saída deixa bastante clara a sua posição: ainda que 
	operem de maneiras distintas, todas tem como objetivo a consolidação 
	de um poder. No sistema penal ``tradicional'', a punição existe 
	praticamente como uma forma de represália dos detentores do poder. 
	Enquanto, nos outros dois sistemas, há um viés em direção da 
	disciplina como garantidora do poder. Grosso modo, é isto. Contudo, 
	quando se fala nos últimos dois modelos, o próprio autor afirma: 
	\newpage % ugh
	\begin{citac}
		As outras se referem, ambas, a uma concepção preventiva, 
		utilitária, corretiva de um direito de punir que pertenceria 
		à sociedade inteira; mas são muito diferentes entre si, ao 
		nível dos dispositivos que esboçam. (FOUCAULT, 1996)
	\end{citac}

	Tratemos, então, destes modelos. O primeiro deles, obra de diversos 
	juristas, tais como Dupaty, Lacratelle e, notavelmente, Beccaria, 
	é praticamente um resultado direto do \textit{zeitgeist} iluminista. 
	O projeto destes reformadores do século XVIII é significativamente 
	mais brando, mais humano. Não obstante, a humanização do sistema 
	penal não é, para Focault, o objetivo destas reformas:  
	``o que vai se definindo não é tanto um respeito novo pela humanidade 
	dos condenados --- os suplícios ainda são frequentes, mesmo para os 
	crimes leves --- quando uma tendência para uma justiça mais 
	desembaraçada e mais inteligente para uma vigilância penal mais atenta 
	ao corpo social.'' %\cite{dp}

	O que, afinal, quer dizer Foucault com isto? Ora, neste primeiro 
	momento, uma justiça ``mais desembaraçada e mais inteligente'' parece 
	confrontar-se com um modelo que era, anteriormente, ou demasiado 
	errático. A justiça flutuava ao sabor do que parecia ser justo ao 
	poder instituído --- o ``rei''. À moda de um pensamento tipicamente 
	iluminista, era preciso regrar este processo, torná-lo previsível. 
	A justiça se converteria num sistema, dotado de lógica, de método, 
	de resoluções. Para Foucault, trata-se meramente de uma forma mais 
	sutil de se introduzir o poder. ``Busca disciplinar-se a mente ao 
	invés de se disciplinar o corpo''. 

	Portanto, o ponto chave aqui é a introdução da disciplina, o que já 
	se prenuncia, de certo modo, quando Foucault confronta, na primeira 
	passagem do livro, uma cena de tortura com um conjunto de regras 
	elaborado para um sistema carcerário. Subentende-se que a violência 
	permanece, a despeito da transição do espetáculo para um sistema 
	mais ``ascético'' de punição. Ela apenas é exercida de uma maneira 
	diferente. O corpo passa a ser submetido a esta \textit{disciplina}, 
	que, em si, não é nada tão novo: ``muitos processos disciplinares 
	existiam há muito tempo: nos conventos, nos exércitos, nas oficinas 
	também. '' Mas que apenas no século XVII e XVIII ela se torna 
	``fórmula geral de dominação''. %p. 170 Ramalhete (PDF/Xerox p. 97)

	Enfim, o elemento mais crucial desta transição entre o sistema penal 
	dito ``tradicional'' e este sistema reformado talvez seja, na visão 
	de Foucault, a conservação de poder, feita de forma limpa, sem a 
	necessidade do espetáculo, da demonstração de poder. O corpo torna-se 
	dócil porque a mente fora ``domesticada'', tal como a mente de um 
	soldado. Mas o que é este poder e no que ele se baseia? 

	A esta altura do campeonato, o eixo do poder muda do paradigma do 
	soberano e súdito para o da administração ``tecnocrática'', por 
	assim dizer. Esta nova forma de poder reside na noção foucaultiana 
	de poder/saber, cuja inspiração é bastante nietzschiana. Em 
	Foucault, \textit{Vigiar e Punir} é apenas o começo do 
	desenvolvimento desta noção como parta da análise política feita 
	pelo autor. Como este poder repousa sobre os elementos estabelecidos 
	aqui, parece fazer sentido a Foucault analisar a sociedade por 
	este prisma. Nos diz J. G. Merquior: 

	\begin{citac}
		The carceral transcends the gaol. The study of the prison, 
		therefore, was bound to unfold into an anatomy of social 
		power at large [...] %\cite merquior, p. 108
	\end{citac}

	E, também: 

	\begin{citac}
		By searching for a genealogy of the modern subject, Foucault 
		was automatically defining an angle where knowledge is 
		enmeshed with power. Thus his pursuit of the modern subject 
		through forms of knowledge as well as practices and discourses 
		had to concentrate on what he calls \textit{power-knowledge}
		(\textit{power-savoir}), a Nietzschean perspective where all 
		will to truth is already a will-to-power. 
	\end{citac}

	Esta nova noção de poder, ainda nascente neste período que Foucault 
	reporta ser o dos reformadores, parece ter contornos bastante 
	científicos, pois baseia-se na observação, na vontade de conquistar 
	a todo o custo uma verdade e dela se apropriar. Em Foucault, até 
	este ponto, ater o conhecimento absoluto parece consubstanciar-se 
	numa forma universal de dominação. E esta interpretação será o norte 
	de Foucault quando trata das novas tecnologias de poder. 
	
	Sobre esta nova forma de poder, nos diz G. Lebrun
	\begin{citac}
		Michel Foucault desenvolveu em \textit{Vigiar e Punir} 
		e n’\textit{A Vontade de Saber}: o poder moderno não é mais, 
		essencialmente, uma instância repressiva e transcendente 
		(o rei acima dos seus súditos, o Estado superior ao indivíduo), 
		mas uma instância de controle, que envolve o \textit{indivíduo 
		mais do que o domina abertamente}. %\cite{lebrun}, p.foda-se* 
	\end{citac} %* No PDF a página é 33, olhar o livro em papel depois

	O projeto reformador combina elementos do primeiro modelo e do 
	terceiro, que está por vir, sob a égide do que chama Foucault de 
	\textit{panoptismo}, se apropriando do conceito benthamiano como 
	símbolo de um novo modelo repressivo. No projeto reformador, o 
	objetivo seria não ainda a repressão, mas a submissão à letra da 
	lei. ``A conjuntura que viu nascer a reforma não é portanto a de 
	uma nova sensibilidade; mas de outra política em relação às 
	ilegalidades''. %\cite{dp}, Ramalhete p.70 (PDF 47esq.)

	\newpage % ;(
	Sobre isto, pontua Merquior: 

	\begin{citac}
		Foucault is clear: at bottom, humanitarianism, in the 
		Enlightenment, counts less than will to power. Underneath 
		noble ideals of human emancipation, the Enlightenment defined 
		new `moral tecnologies' conductive to a degree of social 
		control far greater than was the case in the traditional 
		societies. \textbf{The penal reformers did not not as much 
		want to punish less as to `punish better'} [...] 
		(Nosso grifo) %\cite{merquior}, p. 90
	\end{citac}
	
	Há, então, o estabelecimento da base deste terceiro modelo, 
	que, ao refinar os mecanismos de controle e filtrar elementos 
	espúrios do modelo reformador, consubstancia-se no que Foucault 
	chama de ``sociedade carcerária''. Nesta etapa, a vontade de 
	poder estende-se, difunde-se pelo corpo social. Não é à toa que 
	Foucault elege o panóptico benthamiano como símbolo deste novo 
	paradigma: há uma constante observação do objeto. O indivíduo 
	está a todo o tempo sendo acompanhado, examinado, vigiado, 
	submetido a checagens de normalidade. 

	Esta ``pervasividade'' do novo modelo carcerário penetra 
	estruturalmente a sociedade. Para Foucault, os elementos 
	definidores desta sociedade que vigia e corrige advém dos 
	quartéis, escolas, conventos, hospitais... Todas estas 
	instituições parecem trabalhar no sentido de exercer um poder 
	que não mais é o de um déspota, mas o de um certo padrão de 
	normalidade. O que ecoa um tanto a crítica empreendida também 
	por Foucault noutra obra (\textit{A História da Sexualidade}), em que 
	denuncia um suposto viés vitoriano na sociedade contemporânea 
	que postularia uma espécie de neopuritanismo. 

	Em \textit{Vigiar e Punir}, Foucault ainda trata o poder como 
	uma força represssora, ainda que não unívoca. Trata-se de uma 
	trama de relações complexas, como nos explica G. Lebrun: 

	\begin{citac}
		(...) É o nome atribuído a um conjunto de relações que 
		formigam por toda à parte na espessura do corpo social 
		(poder pedagógico, pátrio poder, poder do policial, poder 
		do contramestre, poder do psicanalista, poder do padre, 
		etc., etc.). %\cite{lebrun} p. foda-se (PDF p. 8)
	\end{citac}

	Por circular que isto possa parecer, é a própria pervasividade 
	do novo modelo que garante sua manutenção através das relações 
	de poder, que passam a ser formatadas por ele. Isto leva o 
	autor a enxergar na ciência (clínica, criminologia, psicologia) 
	elementos constituidores do controle social. 

	Logo, parece bastante seguro concluir que, considerada a visão de 
	nosso autor, isto ruma para uma sociedade disciplinada, técnica. 
	Enfim, corpos dóceis. Desta maneira, ao invés de punir duramente 
	o infrator, esta robótica sociedade busca reprimir o crime diretamente 
	em sua gênese. Se outrora se enfrentava fleumaticamente o crime 
	de maneira reativa, neste momento a racionalidade tomar o leme e 
	prescreve, de acordo com seus ditames, o modo ideal de se agir. 

	Cita-se Foucault, ainda outra vez: 

	\begin{citac}
		O que agora é imposto à justiça penal como o seu ponto de 
		aplicação (...) não será mais o corpo do culpado levantado 
		contra o corpo do rei; não será tampouco o sujeito de 
		direito de um contrato ideal; mas o indivíduo disciplinar. 
		%\citac{foucault} Ramalhete, p. 187 (PDF p. 105)
	\end{citac}

	O homem vê a individualidade ganha durante o período iluminista 
	se dissolver dentre os padrões de normalidade. O indivíduo aqui só 
	existiria mesmo como a menor unidade controlável possível, e porquanto 
	um \textit{indivíduo disciplinar}. 

	Esta perspectiva, entretanto, é bastante complicada. Ela repousa 
	sobre a perspectiva da ``microfísica do poder'', sobre a qual 
	G. Lebrun faz ressalvas diversas ressalvas\footnote{cf. Lebrun, p. 8}. 
	A ideia geral é que este ``adestramento'' denunciado por Foucault 
	até pode ser identificado em diversas sociedades, mas não com a 
	universalidade por ele postulada. Segundo Lebrun, esta é uma 
	contingência do homem europeu, sobretudo, mas não do ``colonizado''. 

	Merquior também aponta, em diversas passagens do capítulo 
	\textit{Charting carceral society} de seu livro \textit{Foucault}, 
	omissões de Foucault tanto de ordem histórica quanto de ordem 
	geográfica\footnote{Merquior, pp. 96-97, pp. 101-107}. O trabalho, 
	enquanto bastante completo no que diz respeito ao tratamento da 
	história penal recente, não consegue abarcar todas as nuances na 
	forma como as reformas penais ocorreram mundo afora. Na América, 
	por exemplo, houve instâncias fortes de rejeição ao modelo 
	panotípico. 

	A Lebrun, ainda, a interpretação de Foucault parece basear-se 
	numa visão que não dá conta da real dimensão do problema político. 
	O professor é categórico: 

	\begin{citac}
		Ele parte, simplesmente, de uma análise sumária e fraudulenta 
		do problema político. Pretende reduzí-lo ao resultado de uma 
		partida: “Indivíduo vs. Estado”. Ora, trata-se de uma partida 
		fraudada. Pois, afinal, o que é este “indivíduo”? De onde 
		provém este átomo social zeloso por seus direitos? Ele já não 
		foi fabricado, sorrateiramente, pelo poder? 
		%\cite{lebrun} p. 34 do PDF
	\end{citac}

	No fim das contas, não seria esta ``sociedade carcerária'' 
	identificada por Foucault uma mera expressão do Leviatã 
	de Hobbes? Afinal, para Hobbes, o estado é frutuo de uma 
	transferência incondicional de poder para as mãos de um 
	indivíduo ou de uma ``assembleia de homens'', que, em troca, 
	confere \textit{segurança} à sociedade que está sob sua égide. 
	Ora, não poderia ser um aumento da disciplina meramente um 
	requerimento da civilização ocidental? 

	Em outras palavras, com um imenso crescimento populacional e 
	uma crescente tendência por parte do homem moderno de esperar 
	do estado alguma proteção, será possível que um sistema mais 
	``relaxado'' de normas daria conta de garantir a tão almejada 
	liberdade? Esta liberdade, como nos diz Lebrun, parece ser 
	apenas quimérica, a este ponto. 

	% COISAS A SE FAZER NESTA PORRA, AINDA

	%-> tratar do terceiro treco

	%-> tentar responder à pergunta da Silvana

	%-> Arrematar com G. Lebrun (p. 34 onwards)

\end{document}
