\documentclass[12pt,a4paper]{article}
% Trabalho de Política I
% Pedro T. R. Pinheiro (8983332)
\usepackage[utf8]{inputenc}
\usepackage[portuguese]{babel}
\usepackage{parskip}
\usepackage{setspace}
\usepackage[left=3.00cm, right=2.00cm, top=3.00cm, bottom=2.00cm]{geometry}
\usepackage{scrextend}

\newenvironment{citac}
{
	\begin{addmargin}
		[4cm]{1em} \footnotesize}{\normalfont 
	\end{addmargin}
	% Para exportar para RTF
%	\footnotesize }{ \normalfont
}

\author{Pedro T. R. Pinheiro\footnote{Aluno pelo departamento de filosofia 
da FFLCH/USP, de número USP 8983332. }}
\title{Foucault e o Cárcere\footnote{
	Trabalho desenvolvido para a matéria FLF0388 - Ética e Filosofia 
	Política I, ministrada pela Profª. Drª. Silvana Ramos. }}
\date{5 de dezembro de 2019}

\begin{document}
	\maketitle
	
	\onehalfspacing
	\setlength{\parskip}{0.5cm}
	
	Noutras circunstâncias, fora mais simples a equação. O ponto é 
	relativamente controverso, mas é preciso assumir, com Michel 
	Foucault, que o direito de punir já foi, um dia, tido como 
	prerrogativa “natural” de um soberano, a fim de iniciar, com 
	perspectivas de sucesso, essa viagem. Em \textit{Vigiar e Punir}, 

	Em Vigiar e Punir, o filósofo postula que a história moderna 
	das formas de punir descortina ao menos três modelos possíveis, 
	e este teria sido o ponto de partida. Com o avanço da técnica e 
	da ciência em tempos de iluminismo, este modelo, expressão do 
	Leviatã em toda sua potência, substitui-se, transforma-se, 
	sofistica-se. Exploraremos neste trabalho um pouco deste desenvolvimento, 
	com ênfase no terceiro e último modelo e, sobretudo, como ele parece 
	ter-se estabelecido como state-of-the-art no que diz respeito ao 
	controle social.
	
	Sobre estes três modelos, compreende-se: 
	
	\begin{citac}
		[...] pode-se dizer que nos encontramos no fim do século XVIII 
		diante de três maneiras de organizar o poder de punir. [...] 
		Esquematizando muito, poderíamos dizer que, no poder 
		monárquico, a punição é um cerimonial da soberania [...]. No 
		projeto dos juristas reformadores, a punição é um processo 
		para requalificar os indivíduos como sujeitos de direito 
		[...]. Enfim, no projeto de instituição carcerária que se 
		elabora, a punição é uma técnica de coerção dos indivíduos. 
		[...] O soberano e sua força, o corpo social, o aparelho 
		administrativo. [...] São três modalidades de acordo com 
		as quais se exerce o poder de punir. Três tecnologias de poder. 
		\cite{foucault}
	\end{citac}

	Ao delinear tal \emph{genealogia} da evolução dos sistemas penais, 
	Foucault já de saída deixa bastante clara a sua posição: ainda que 
	operem de maneiras distintas, todos têm como objetivo a consolidação 
	de um poder. No sistema penal ``tradicional'', a punição existe 
	praticamente como uma forma de represália dos detentores do poder. 
	Enquanto, nos outros dois sistemas, há um viés em direção da 
	disciplina como garantidora do poder. Grosso modo, é isto. Contudo, 
	no que diz respeito às duas últimas formas de punir, o 
	próprio autor afirma: 
	\begin{citac}
		As outras se referem, ambas, a uma concepção preventiva, 
		utilitária, corretiva de um direito de punir que pertenceria 
		à sociedade inteira; mas são muito diferentes entre si, ao 
		nível dos dispositivos que esboçam. \cite{foucault}
	\end{citac}

	Tratemos, então, desses modelos. O primeiro deles, obra de diversos 
	juristas, tais como Dupaty, Lacratelle e, notavelmente, Beccaria, 
	é praticamente um resultado direto do \textit{zeitgeist} iluminista. 
	O projeto destes reformadores do século XVIII é significativamente 
	mais brando, mais humano. Não obstante, a humanização do sistema 
	penal não é, para Focault, o objetivo destas reformas:  
	``o que vai se definindo não é tanto um respeito novo pela humanidade 
	dos condenados --- os suplícios ainda são frequentes, mesmo para os 
	crimes leves --- quanto uma tendência para uma justiça mais 
	desembaraçada e mais inteligente para uma vigilância penal mais atenta 
	ao corpo social.'' 

	O que, afinal, quer dizer Foucault com isso? Ora, neste primeiro 
	momento, uma justiça ``mais desembaraçada e mais inteligente'' parece 
	confrontar-se com um modelo que era, anteriormente, ou demasiadamente 
	impróprio e desvantajoso. A justiça flutuava ao sabor do que parecia ser 
	justo ao poder instituído --- o ``rei''. À moda de um pensamento 
	tipicamente iluminista, era preciso regrar este processo, torná-lo 
	previsível. A justiça se converteria num sistema, dotado de lógica, de 
	método, de resoluções. Para Foucault, trata-se meramente de uma forma 
	mais sutil de se introduzir o poder. ``Busca disciplinar-se a mente ao 
	invés de se disciplinar o corpo''. 

	O ponto chave aqui é a introdução da disciplina, o que já 
	se prenuncia, de certo modo, quando Foucault confronta, na primeira 
	passagem do livro, uma cena de tortura com um conjunto de regras 
	elaborado para um sistema carcerário. Subentende-se que a violência 
	permanece, a despeito da transição do espetáculo para um sistema 
	mais ``ascético'' de punição. Ela apenas é exercida de uma maneira 
	diferente. O corpo passa a ser submetido a esta \textit{disciplina}, 
	que, em si, não é nada tão novo: ``muitos processos disciplinares 
	existiam há muito tempo: nos conventos, nos exércitos, nas oficinas 
	também. '' Mas é apenas no século XVII e XVIII que ela se torna 
	``fórmula geral de dominação''. \cite{foucault} (p. 170)

	Enfim, o elemento mais crucial desta transição entre o sistema penal 
	dito ``tradicional'' e este sistema reformado talvez seja, na visão 
	de Foucault, a conservação de poder, feita de forma limpa, sem a 
	necessidade do espetáculo, da demonstração de poder. O corpo torna-se 
	dócil porque a mente fora ``domesticada'', tal como a mente de um 
	soldado. Mas o que é este poder e em que ele se baseia? 

	A esta altura do campeonato, o eixo do poder muda do paradigma do 
	soberano e súdito para o da administração ``tecnocrática'', por 
	assim dizer. Esta nova forma de poder reside na noção foucaultiana 
	de poder/saber, cuja inspiração é notoriamente nietzschiana. Em 
	Foucault, ``vigiar'' e ``punir'' é apenas o começo do 
	dessa noção como parte da análise política desenolvida no livro. 
	Como este poder repousa sobre os elementos estabelecidos 
	aqui, a Foucault, parece fazer sentido analisar a sociedade por 
	este prisma. Em seu corrosivo livro \textit{Foucault}, diz-nos 
	J. G. Merquior: 

	\begin{citac}
		O carcerário transcende o cárcere. Por conseguinte, o estudo 
		da prisão teria fatalmente de se desdobrar numa anatomia do poder 
		social em geral [...] \cite{merquior} (p. 165)
	\end{citac}

	E, também: 

	\begin{citac}
		Ao procurar uma genealogia do sujeito moderno, Foucault estava 
		automaticamente definindo um ângulo em que o saber está entrelaçado 
		com o poder. Assim, sua investigação do sujeito moderno, por meio das 
		formas de saber, bem como de práticas e discursos, tinha de se 
		concentrar no que ele chama de \textit{poder-saber} (\textit
		{pouvoir-savoir}), uma perspectiva nietzschiana em que toda a verdade 
		já constitui uma vontade de poder. [...] \cite{merquior} (p.165-166)
	\end{citac}

	Esta nova noção de poder, ainda nascente neste período que Foucault 
	reporta como o dos reformadores, parece ter contornos bastante 
	científicos, pois baseia-se na observação, na vontade de conquistar 
	a todo o custo uma verdade e dela se apropriar. Em Foucault, até 
	este ponto, ater o conhecimento absoluto parece consubstanciar-se 
	numa forma universal de dominação. E esta interpretação será o norte 
	de Foucault quando trata das novas tecnologias de poder. 
	Não obstante, sobre ela, afirma G. Lebrun: 
	
	\begin{citac}
		Michel Foucault desenvolveu em \textit{Vigiar e Punir} 
		e n’\textit{A Vontade de Saber}: o poder moderno não é mais, 
		essencialmente, uma instância repressiva e transcendente 
		(o rei acima dos seus súditos, o Estado superior ao indivíduo), 
		mas uma instância de controle, que envolve o \textit{indivíduo 
		mais do que o domina abertamente}. \cite{lebrun} (p. 70) 
	\end{citac}

	O projeto reformador combina elementos do primeiro modelo e do 
	terceiro, que está por vir, sob a égide do que chama Foucault de 
	\textit{panoptismo}, se apropriando do conceito benthamiano como 
	símbolo de um novo modelo repressivo. No projeto reformador, o 
	objetivo seria não ainda a repressão, mas a submissão à letra da 
	lei. ``A conjuntura que viu nascer a reforma não é portanto a de 
	uma nova sensibilidade; mas de outra política em relação às 
	ilegalidades''. \cite{foucault} (p. 70)
    Sobre isto, pontua Merquior: 

	\begin{citac}
		Foucault é claro: no fundo, o humanitarismo, o iluminismo, 
		contava menos que a vontade de poder. Por baixo de seus nobres 
		ideais de emancipação humana, o iluminismo defininia novas 
		``tecnologias morais'', conducentes a um grau de controle social 
		muito maior do que o existente nas sociedades tradicionais.
		\textbf{Mais do que punir menos, os reformadores 
		sociais desejavam `punir melhor'} (grifo nosso). 
		\cite{merquior} (p. 129)
	\end{citac}
	
	Teria lugar, então, o estabelecimento da base deste terceiro modelo, 
	que, ao refinar os mecanismos de controle e filtrar elementos 
	espúrios do modelo reformador, consubstancia-se no que Foucault 
	chama de ``sociedade carcerária''. Nesta etapa, a vontade de 
	poder estende-se, difunde-se pelo corpo social. Não é à toa que 
	Foucault elege o panóptico benthamiano como símbolo deste novo 
	paradigma: há uma constante observação do objeto. O indivíduo 
	está a todo o tempo sendo acompanhado, examinado, vigiado, 
	submetido a checagens de normalidade. 

	Esta ``pervasividade'' do novo modelo carcerário penetra 
	estruturalmente a sociedade. Para Foucault, os elementos 
	definidores desta sociedade que vigia e corrige advém dos 
	quartéis, escolas, conventos, hospitais... Todas estas 
	instituições parecem trabalhar no sentido de exercer um poder 
	que não mais é o de um déspota, mas o de um certo padrão de 
	normalidade. O que ecoa um tanto a crítica empreendida também 
	por Foucault noutra obra (\textit{A História da Sexualidade}), em que 
	denuncia um suposto viés vitoriano na sociedade contemporânea 
	que postularia uma espécie de neopuritanismo. 

	Em \textit{Vigiar e Punir}, Foucault também caracteriza o poder como 
	uma força represssora, ainda que não unívoca. Trata-se de uma 
	trama de relações complexas, como nos explica G. Lebrun: 

	\begin{citac}
		(...) É o nome atribuído a um conjunto de relações que 
		formigam por toda à parte na espessura do corpo social 
		(poder pedagógico, pátrio poder, poder do policial, poder 
		do contramestre, poder do psicanalista, poder do padre, 
		etc., etc.). \cite{lebrun} (pp. 20-21)
	\end{citac}

	Por circular que isto possa parecer, é a própria pervasividade 
	do novo modelo que garante sua manutenção através das relações 
	de poder, que passam a ser formatadas por ele. Isto leva o 
	autor a enxergar na ciência (clínica, criminologia, psicologia) 
	elementos constituidores do controle social. 

	Logo, parece bastante seguro concluir que, do ponto de vista do nosso 
	autor, tudo isso conduz à estruturação de uma sociedade controlada, 
	disciplinada, técnica. Enfim, reeducada com os seus corpos dóceis 
	para a vivência do capitalismo. Desta maneira, ao invés de punir duramente 
	o infrator, esta robótica sociedade busca reprimir o crime diretamente 
	em sua gênese. Se outrora se enfrentava fleumaticamente o crime 
	de maneira reativa, neste momento a racionalidade tomar o leme e 
	prescreve, de acordo com seus ditames, o modo ideal de se agir. 
	Ou, como pontifica Foucault: 

	\begin{citac}
		O que agora é imposto à justiça penal como o seu ponto de 
		aplicação (...) não será mais o corpo do culpado levantado 
		contra o corpo do rei; não será tampouco o sujeito de 
		direito de um contrato ideal; mas o indivíduo disciplinar. 
		\cite{foucault} (p. 187)
	\end{citac}

	E seria desse  modo que o homem veria individualidade, adquirida durante
	o período iluminista, dissolver-se entre os padrões de normalidade. 
	O indivíduo aqui só existiria mesmo como a menor unidade controlável 
	possível, e porquanto um \textit{indivíduo disciplinar}. 

	Esta condição, entretanto, não é fácil de estabelecer. Ela repousa 
	sobre a perspectiva da ``microfísica do poder'', sobre a qual 
	G. Lebrun faz diversas ressalvas\footnote{cf. Lebrun, p. 8}. 
	A ideia geral é que este ``adestramento'' denunciado por Foucault 
	até pode ser identificado em diversas sociedades, mas não com a 
	universalidade por ele postulada. Segundo Lebrun, esta é uma 
	contingência do homem europeu, sobretudo, mas não do ``colonizado''. 

	Merquior também aponta, em diversas passagens do capítulo 
	\textit{Charting carceral society} de seu já mencionado livro, 
	omissões de Foucault tanto de ordem histórica quanto de ordem 
	geográfica\footnote{Merquior, pp. 96-97, pp. 101-107}. O trabalho, 
	enquanto bastante completo no que diz respeito ao tratamento da 
	história penal recente, não consegue abarcar todas as nuances na 
	forma como as reformas penais ocorreram mundo afora. Na América, 
	por exemplo, houve instâncias fortes de rejeição ao modelo 
	panotípico. 

	Ainda na óptica de Lebrun, a interpretação de Foucault parece basear-se 
	numa visão que não dá conta da real dimensão do problema político. 
	O professor é categórico: 

	\begin{citac}
		Ele parte, simplesmente, de uma análise sumária e fraudulenta 
		do problema político. Pretende reduzi-lo ao resultado de uma 
		partida: “Indivíduo vs. Estado”. Ora, trata-se de uma partida 
		fraudada. Pois, afinal, o que é este “indivíduo”? De onde 
		provém este átomo social zeloso por seus direitos? Ele já não 
		foi fabricado, sorrateiramente, pelo poder? 
		\cite{lebrun} (p. 87)
	\end{citac}

	Se o foi — se é verdade que o próprio poder”, quer pelos sucessos quer 
	pelos fracassos dos seus projetos espúrios, transformou o indivíduo, de 
	vítima do Leviatã, neste “átomo social zeloso por seus direitos” —, então 
	é preciso voltar àquele ponto de partida. E perguntar se, de fato, algum 
	dia, teria existido mesmo um estado capaz de vigiar e punir sem a 
	autorização dos súditos passíveis de punição. 
	
    Ou se, pelo contrário, como parece nos ensinar Hobbes, não seríamos nós 
    mesmos — aqueles indivíduos que transferem incondicionalmente o poder às 
    mãos de um só indivíduo ou de uma “assembleia de homens” em troca de 
    proteção e segurança — os requerentes, cada vez mais exigentes, de um 
    aumento de racionalidade e disciplina no interior da civilização ocidental. 
    Nesse caso, como propõe Lebrun, a esperança de um sistema mais relaxado de
    normas pode nunca passar de uma quimera. 

    %\begin{thebibliography}{9}
	%	\bibitem{adorno}
	%	ADORNO, T. 
	%	\textit{Education after Auschwitz. }   
	%	\\\texttt{https://www.ime.usp.br/~tadeu/EDF0285/A10\_Adorno.pdf}
	%	
	%	
	%\end{thebibliography}
	
	\vspan
	
	\bibliographystyle{apalike}
	\bibliography{trab}

\end{document}
