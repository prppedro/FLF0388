\documentclass[12pt,a4paper]{article}
% Trabalho de Política I
% Pedro T. R. Pinheiro (8983332)
\usepackage[utf8]{inputenc}
\usepackage[portuguese]{babel}
\usepackage{parskip}
\usepackage{setspace}
\usepackage[left=3.00cm, right=2.00cm, top=3.00cm, bottom=2.00cm]{geometry}
\usepackage{scrextend}

\newenvironment{citac}
{
	\begin{addmargin}
		[4cm]{1em} \footnotesize}{\normalfont 
	\end{addmargin}
	% Para exportar para RTF
%	\footnotesize }{ \normalfont
}

\author{Pedro T. R. Pinheiro\footnote{Aluno pelo departamento de filosofia 
da FFLCH/USP, de número USP 8983332. }}
\title{Foucault e o Corpo\footnote{
	Trabalho desenvolvido para a matéria FLF0388 - Ética e Filosofia 
	Política I, ministrada pela Profª. Drª. Silvana Ramos. }}
\date{25 de janeiro de 2019}

\begin{document}
	\maketitle
	
	\onehalfspacing
	\setlength{\parskip}{0.5cm}
	
	Entre 1970 e 1975, M. Foucault regidigia uma de suas obras mais 
	consagradas: \textit{Vigiar e Punir}. Para alguns, uma de suas 
	obras mais importantes. Em \textit{Vigiar e Punir}, Foucault faz 
	um detalhado exame das formas de punição e como a evolução delas 
	parece estar intrinsicamente conectada à forma como o poder dominante 
	se faz sentir. Em seu mais refinado nível, o poder se disfarçaria 
	como uma força que não mata ou pune, mas sim disciplina. Disciplina 
	esta que não se limita mais às prisões, estando, segundo o autor, 
	presentes em conventos, escolas, hospitais e afins. 
	
	\textit{Vigiar e Punir} introduz conceitos que serão basilares 
	dentro do \textit{opus} foucaultiano. Quando esboçada a perspectiva 
	de ``corpos dóceis'', tem-se um prelúdio do que Foucault toma como 
	principal ponto em sua obra subsequente, \textit{A História da 
	Sexualidade}. Esta transição é perfeitamente descrita pelo seguinte 
	trecho: 
	
	\begin{citac}
	A velha potência de morte em que se simbolizava o poder soberano 
	é agora, cuidadosamente, recoberta pela administração dos corpos e pela 
	gestão calculista da vida. Desenvolvimento rápido, no decorrer da época 
	clássica, das disciplinas diversas --- escolas, colégios, casernas, 
	ateliês; aparecimento, também, no terreno das práticas políticas e 
	observações econômicas dos problemas de natalidade, longevidade e 
	saúde pública, habitação e migração; explosão, portanto, de técnicas 
	diversas e numerosas para obterem a sujeição dos corpos e o controle 
	das populações. Abre-se, assim, a era de um ``biopoder''
	(\cite{hs}, pp. 131-132)
	\end{citac}
	
	\textit{A História da Sexualidade} continua de onde {Vigiar e Punir} 
	``parou'', portanto. É, também, uma investigação histórica, mas com 
	outro foco: como, paralelamente aos processos descritos na obra anterior, 
	o poder dominante criou formas de submeter o corpo a um rígido regime, 
	que, posteriormente, até facilitaria a criação destes ``corpos dóceis.''
	
	\newpage
	
	Ora, da mesma forma em que, em \textit{Vigiar e Punir}, as execuções 
	públicas, tortura e humilhação foram substituídas por um ascético e, em 
	teoria, preciso instrumento de aplicação ou execução da lei, e o carrasco 
	substituído por médicos, capelões etc., aqui há, em paralelo, uma 
	transição entre esta ``potência da morte'' e a ``gestão calculista da 
	vida''. O veículo, entretanto, escolhido pelo autor para tratar destas 
	transições é a sexualidade. É uma estranha escolha. Foucault, entretanto, 
	a justifica da seguinte maneira: 
	
	\begin{citac}
		[A sexualidade] aparece mais como um ponto de passagem particularmente 
		denso pelas relações de poder; entre homens e mulheres, entre jovens e 
		velhos, entre pais e filhos, entre educadores e alunos, entre padres e 
		leigos, entre administração e população. Nas relações de poder, a 
		sexualidade não é o elemento mais rígido, mas um dos dotados da maior 
		instrumentalidade: utilizável no maior número de manobras, podendo 
		servir de ponto de apoio, de articulação às mais variadas estratégias. 
		(\cite{hs}, p. 98)
	\end{citac}
	
	Foucault, logo, parece sugerir que por um certo caráter ``universal'' da 
	sexualidade
	\footnote{``A meta cofessa [...] é destacar o \textit{discurso} do sexo 
	em relação às `tecnologias' polimorfas do poder. Não o sexo como prática, 
	mas o sexo como tema de uma prática discursiva multiforme (...)'' 
	\cite{merquior}, pp. 183-184}, 
	ela pode ser explorada e manipulada ideologicamente a fim de 
	transportar para as mais profundas camadas do indivíduo o ethos burguês, 
	puritano. Seria o ponto de partida da estrada da repressão sexual. Nós, 
	``outros vitorianos'', assimilamos de maneira tão forte esta manipuação 
	que ainda hoje o sexo é visto como sujo, ou, no mínimo, um assunto tabu. 
	A repressão, portanto, prevalece: 
	
	\begin{citac}
		Esse discurso sobre a repressão moderna do sexo se sustenta. Sem 
		dúvida porque é fácil de ser dominado. Uma grave caução histórica e 
		política o protege; pondo a origem da Idade da Repressão no século 
		XVII, após centenas de anos de arejamento e de expressão livre, 
		faz-se com que coincida com o desenvolvimento do capitalismo: ela 
		faria parte da ordem burguesa. 
		(\cite{hs}, p. 11)
	\end{citac}
	
	As tecnologias de poder, estão, então, numa cômoda posição a partir da 
	qual podem, com respaldo ideologico, operar o corpo e discipliná-lo, 
	sem que sequer seja preciso tocá-lo. É o modelo ``calculista'' de gestão 
	da vida, que prescreve, às vezes com considerável autoridade científica, 
	cursos de ação que o indivíduo adota por conta própria. Tratar-se-ia de 
	um mecanismo autoimposto de repressão e censura. 
	
	Até este ponto, os dois livros parecem estar nos mesmos trilhos. Em 
	\textit{A História da Sexualidade}, Foucault ``estende seu argumento''
		\footnote{cf. \cite{pwkn}, p. 99}
	para demonstrar, dentre outras coisas, como, por exemplo, a confissão 
	tomou parte ativa e integral na constituição desta sociedade que se 
	``auto-reprimiria'': 
	
	\begin{citac}
		[...] nos tornamos uma sociedade singularmente confessanda. A 
		confissão difundiu amplamente seus efeitos: na justiça, na medicina, 
		na pedagogia, nas relações familiares, nas relações amorosas, na 
		esfera mais cotidiana e nos ritos mais solenes; confessam-se os crimes, 
		os pecados [...] confessa-se em público, em particular, aos pais, aos 
		educadores, ao médico, àqueles quem se ama [...] Confessa-se --- ou 
		é forçado a confessar. Quando a confissão não é espontânea ou imposta 
		por algum imperativo interior, é extorquida [...]
		(\cite{hs}, p. 59)		
	\end{citac}

	Esta ``sociedade confessanda'', entretanto, raramente precisa ter a 
	verdade ``extorquida'', pois os ``imperativos interiores'' a fariam 
	--- em nome da ciência, segurança, saúde ---, por conta própria e de 
	boa vontade, não só confessar, mas admitir que o material da confissão 
	tem foro específico e que, portanto, o objeto da confissão tem de ficar 
	confinado dentro dos domínios que lhe cabem. 
	
	Que será este ``corpo dócil'' senão um corpo que de própria intenção se 
	disciplinou, se aplicou? Neste ponto, duas coisas explicam os corpos 
	dóceis: a pervasividade da última tecnologia de poder apresentada em 
	\textit{Vigiar e Punir}
	\footnote{``Enfim, no projeto de instituição carcerária que se elabora, 
	a punição é uma técnica de coerção dos indivíduos...'' 
	\cite{foucault}, pp. 115-116}
	e a instituição, ao longo dos séculos, de tecnologias de controle do 
	``eu'', constituídas no âmago de instituições eclesiásticas, 
	administrativas e familiares. 
	
	As ``tecnologias do eu'' constituem em importante tema, em \textit{A 
	História da Sexualidade}, demarcando um retorno por parte de Foucault 
	à uma \emph{genealogia do sujeito}
		\footnote{cf. \cite{merquior}, p. 183}, 
	o que é mais compatível com sua ideia de que ``o corpo e tudo o que 
	o toca pertence ao domínio da genealogia''
		\footnote{cf. \cite{map}, p. 35}. 
	A partir deste momento, Foucault adota uma perspectiva que lembra menos 
	sua postura ainda bastante ``arqueológica'' em \textit{Vigiar e Punir} 
	e se aventura a explorar a questão do biopoder a partir do ponto de 
	vista do corpo. A dominação é vista como algo interiorizado no \emph{eu}. 
	
	Em sua versão mais refinada, esta dominação não é mais a adaga no pescoço 
	do servo ou lança na barriga do escravo. Todos, em tese, se sujeitariam, 
	à esta nova forma. Quem cria e perpetra a dominação, torna-se, no fim das 
	contas, tão dominado quanto que pretendia-se dominar. Esta hipótese 
	repressiva de um estado onipresente, opulento, embora rechaçada por alguns 
	autores
		\footnote{cf. \cite{lebrun}, p. 8}, 
	torna-se uma figura sempre presente como pano de fundo dos escritos 
	foucaultianos. E, ao indivíduo, a esta altura do campeonato, ela 
	apareceria sob o disfarce de um estado de direito cuja lei se aplicaria 
	igualmente a todos. 
	
	Em nome da precisão, é preciso também que se retifique o sentido dado 
	ao termo ``repressão'', pois neste contexto o sentido é bastante diferente 
	daquele dado pelos teóricos da escola de Frankfurt, como, por exemplo, 
	H. Marcuse 
		\footnote{cf. \cite{merquior}, p. 187}. 
	É importante frisar que os mesmos mecanismos que produzem a confissão 
	também produzem a sexualidade da forma como ela se manifesta. À luz de 
	eras anteriores, pode aparecer hoje de forma mais reprimida. Mas, para 
	Foucault, assim se manifesta por ter sido produzida deste modo pelo 
	discurso vigente. A sexualidade, da forma como Foucault sugere que ela 
	se apresentaria, é mais resultado de uma formatação, portanto, do que 
	uma série de atividades repressivas, ao ponto de que as entidades 
	detentoras do poder não mais precisaria temer o sexo --- ele já está 
	sob controle discursivo. 
	
	Contudo, como se apresenta esta situação nestes tempos, especialmente 
	após a contra-cultura? A partir da década de 60, começou a se pregar uma 
	libertação deste discurso. Para Foucault, isto é de pouco efeito, pois, 
	num nível superior, o sexo ainda soa como assunto tabu. De qualquer modo, 
	a impressão que se tem na contemporaneidade é que nunca se falou tanto 
	sobre sexo: há ``sexólogos'' falando na TV, ``coachs tântricos'', 
	publicações oficiais. É um assunto corrente na mídia. Está também nas 
	piadas --- embora, é verdade, frequentemente aproximado com cautela, 
	velado por entre metáforas e subterfúgios. 
	
	Estas metáforas, eufemismos, subterfúgios, podem, sim, ser resquícios do 
	discurso que produz esta sexualidade da qual resulta o corpo disciplinado, 
	mas, dia após dia, parece ter menos e menos efeito. Após a detecção de 
	distúrbios provocados pela sexualidade não discutida abertamente, bem 
	como a rápida proliferação de doenças sexualmente transmissíveis (tal como 
	a síndrome de imuno-deficiência adquirida --- AIDS --- nos anos 80), 
	parece ganhar cada vez mais força a ideia de que a sexualidade tem de ser 
	discutida abertamente em diversos foros --- como nas escolas, por exemplo. 
	
	

	\newpage
	\bibliographystyle{apalike}
	\bibliography{trab}

\end{document}
