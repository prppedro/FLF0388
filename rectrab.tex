\documentclass[12pt,a4paper]{article}
% Trabalho de Política I
% Pedro T. R. Pinheiro (8983332)
\usepackage[utf8]{inputenc}
\usepackage[portuguese]{babel}
\usepackage{parskip}
\usepackage{setspace}
\usepackage[left=3.00cm, right=2.00cm, top=3.00cm, bottom=2.00cm]{geometry}
\usepackage{scrextend}

\newenvironment{citac}
{
	\begin{addmargin}
		[4cm]{1em} \footnotesize}{\normalfont 
	\end{addmargin}
	% Para exportar para RTF
%	\footnotesize }{ \normalfont
}

\author{Pedro T. R. Pinheiro\footnote{Aluno pelo departamento de filosofia 
da FFLCH/USP, de número USP 8983332. }}
\title{Foucault e o Corpo\footnote{
	Trabalho desenvolvido para a matéria FLF0388 - Ética e Filosofia 
	Política I, ministrada pela Profª. Drª. Silvana Ramos. }}
\date{25 de janeiro de 2019}

\begin{document}
	\maketitle
	
	\onehalfspacing
	\setlength{\parskip}{0.5cm}
	
	Entre 1970 e 1975, M. Foucault regidigia uma de suas obras mais 
	consagradas: \textit{Vigiar e Punir}. Para alguns, uma de suas 
	obras mais importantes. Em \textit{Vigiar e Punir}, Foucault faz 
	um detalhado exame das formas de punição e como a evolução delas 
	parece estar intrinsicamente conectada à forma como o poder dominante 
	se faz sentir. Em seu mais refinado nível, o poder se disfarçaria 
	como uma força que não mata ou pune, mas sim disciplina. Disciplina 
	esta que não se limita mais às prisões, estando, segundo o autor, 
	presentes em conventos, escolas, hospitais e afins. 
	
	\textit{Vigiar e Punir} introduz conceitos que serão basilares 
	dentro do \textit{opus} foucaultiano. Quando esboçada a perspectiva 
	de ``corpos dóceis'', tem-se um prelúdio do que Foucault toma como 
	principal ponto em sua obra subsequente, \textit{A História da 
	Sexualidade}. Esta transição é perfeitamente descrita pelo seguinte 
	trecho: 
	
	\begin{citac}
	“A velha potência de morte em que se simbolizava o poder soberano 
	é agora, cuidadosamente, recoberta pela administração dos corpos e pela 
	gestão calculista da vida.” (\cite{hs}, p. ???????)
	\end{citac}
	
	\textit{A História da Sexualidade} continua de onde {Vigiar e Punir} 
	``parou'', portanto. É, também, uma investigação histórica, mas com 
	outro foco: como, paralelamente aos processos descritos na obra anterior, 
	o poder dominante criou formas de submeter o corpo a um rígido regime, 
	que, posteriormente, até facilitaria a criação destes ``corpos dóceis?''


	\bibliographystyle{apalike}
	\bibliography{trab}

\end{document}
